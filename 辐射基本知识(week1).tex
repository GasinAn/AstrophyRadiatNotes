\chapter{辐射基本知识}

\section{高斯单位制}

国际单位制: $\text{m}$, $\text{kg}$, $\text{s}$; $\mu_0:=4\pi\times10^{-7}\text{N}/\text{A}^2$.

Gauss单位制: $\text{cm}$, $\text{g}$, $\text{s}$;
\begin{enumerate}
    \item 令$\epsilon_0=\mu_0=1$\footnote{国际制中$c=\frac{1}{\sqrt{\epsilon_0\mu_0}}$, Gauss制中$c=\frac{c}{\sqrt{\epsilon_0\mu_0}}$.},
    \item 令库伦定律中比例系数为1, 得$q$单位$\text{cm}^{3/2}\text{g}^{1/2}\text{s}^{-1}$,
    \item 令$\b{B}$单位和$\b{E}$单位相同.
\end{enumerate}
直接计算方法: 把公式看成数的等式而非量的等式\footnote{请自行咨询梁老师这句话的含义.}.
\begin{table}[htbp]
    \centering
    \begin{tabular}{|c|c|}
        \hline
        物理量 & 单位定义式\\
        \hline
        \hline
        $q$ & $F=\frac{q_1q_2}{r^2}$\\
        \hline
        \hline
        $\epsilon_0$ & $\epsilon_0=1$\\
        \hline
        $\epsilon$ & $\epsilon=\epsilon_\text{r}\epsilon_0$\\
        \hline
        $E$ & $F=qE$\\
        \hline
        $p$ & $p=ql$\\
        \hline
        $P$ & $P=\frac{\sum p}{\Delta V}$\\
        \hline
        $D$ & $D=\epsilon E$\\
        \hline
        $\chi_\text{e}$ & $P=\chi_\text{e}E$\\
        \hline
        \hline
        $I$ & $I=\frac{\d q}{\d t}$\\
        \hline
        $U$ & $U=Ed$\\
        \hline
        $R$ & $U=IR$\\
        \hline
        $\mathcal{E}$ & $\mathcal{E}=U+IR$\\
        \hline
        $C$ & $C=\frac{q}{U}$\\
        \hline
        $L$ & $L=\mathcal{E}\frac{\d I}{\d t}$\\
        \hline
        \hline
        $\mu_0$ & $\mu_0=1$\\
        \hline
        $\mu$ & $\mu=\mu_\text{r}\mu_0$\\
        \hline
        $B$ & $F=q\frac{v}{c}B$\\
        \hline
        $m$ & $m=\frac{I}{c}S$\\
        \hline
        $M$ & $M=\frac{\sum m}{\Delta V}$\\
        \hline
        $H$ & $B=\mu H$\\
        \hline
        $\chi_\text{m}$ & $M=\chi_\text{m}H$\\
        \hline
    \end{tabular}
    \caption{Gauss单位制}
\end{table}
