\chapter{辐射场理论}

\section{有源场: 电磁势---求解辐射场的经典方法}

令$\partial^a A_a(t,\b{r}) = 0$, 得
\begin{equation}
    \partial^a \partial_a A_b(t,\b{r}) = -4\pi J_b(t,\b{r}).
\end{equation}
有解
\begin{equation}
    A^a(t,\b{r}) = \frac{1}{c} \int 
    \frac{J^a(t-\frac{\left\lvert\b{r}-\b{r}'\right\rvert}{c},\b{r}')}
    {\left\lvert\b{r}-\b{r}'\right\rvert}\,\d V'.
\end{equation}

对单粒子, $J^a(t,\b{r}) = q U^a(t) \delta(\b{r}-\b{r}_q(t))$, 所以
\begin{align}
    A^a(t,\b{r}) &= \frac{q}{c} \int 
    \frac{U^a(t-\frac{\left\lvert\b{r}-\b{r}'\right\rvert}{c})
    \delta(\b{r}'-\b{r}_q(t-\frac{\left\lvert\b{r}-\b{r}'\right\rvert}{c}))}
    {\left\lvert\b{r}-\b{r}'\right\rvert}\,\d V'\\
    &= \frac{q}{c} \int 
    \frac{ \int
    U^a(t')
    \delta(\b{r}'-\b{r}_q(t'))
    \delta(t'-(t-\frac{\left\lvert\b{r}-\b{r}'\right\rvert}{c}))\,\d t'
    }
    {\left\lvert\b{r}-\b{r}'\right\rvert}\,\d V'\\
    &= \frac{q}{c} \iint 
    \frac{
    U^a(t')
    \delta(t'-(t-\frac{\left\lvert\b{r}-\b{r}'\right\rvert}{c}))
    }
    {\left\lvert\b{r}-\b{r}'\right\rvert}
    \delta(\b{r}'-\b{r}_q(t'))\,\d V'\,\d t'\\
    &= \frac{q}{c} \int 
    \frac{
    U^a(t')
    \delta(t'-(t-\frac{\left\lvert\b{r}-\b{r}_q(t')\right\rvert}{c}))
    }
    {\left\lvert\b{r}-\b{r}_q(t')\right\rvert}\,\d t',
\end{align}
上式是对$t'$进行的积分. 把积分变量变成$\tilde{t}'=t'-(t-\frac{\left\lvert\b{r}-\b{r}_q(t')\right\rvert}{c})$.
注意到积分中$t$和$\b{r}$被当成常量, 所以$\d\tilde{t}'=\d t'+\frac{\dot{R}(t')}{c}\d t'=(1+\frac{\dot{R}(t')}{c})\d t'$,
其中$R(t')=\left\lvert\b{r}-\b{r}_q(t')\right\rvert$, 所以$\dot{R}(t')=-\frac{(\b{r}-\b{r}_q(t'))}{\left\lvert\b{r}-\b{r}_q(t')\right\rvert}\cdot\dot{\b{r}}_q(t')=-\hat{\b{n}}(t')\cdot\b{u}_q(t')$.

定义$K(t')=1+\frac{\dot{R}(t')}{c}=1-\frac{\hat{\b{n}}(t')\cdot\b{u}_q(t')}{c}=1-\frac{u_r(t')}{c}$,
则$\d\tilde{t}'=K(t')\d t'$, 所以
\begin{align}
    A^a(t,\b{r})
    &= \frac{q}{c} \int 
    \frac{U^a(t')}
    {K(t')\left\lvert\b{r}-\b{r}_q(t')\right\rvert}\delta(\tilde{t}')\,\tilde{t}',
\end{align}
其中$t'$是$\tilde{t}'$和$t$, $\b{r}$的函数. 令$\tilde{t}'=t'-(t-\frac{\left\lvert\b{r}-\b{r}_q(t')\right\rvert}{c})=0$,
可得$R(t')=\left\lvert\b{r}-\b{r}_q(t')\right\rvert=c(t-t')$, 所以
\begin{equation}
    A^a(t,\b{r}) = \left[\frac{qU^a(t')}{cK(t')R(t')}\right]_{R(t')=c(t-t')}.
\end{equation}
$t$时刻$\b{r}$处接收到$t'$时刻$r_q(t')$处产生的辐射, 恰有$R(t')=c(t-t')$. 给定$t$和$\b{r}$, 满足$R(t')=\left\lvert\b{r}-\b{r}_q(t')\right\rvert=c(t-t')$的$t'$存在且唯一, 否则世界线不类时.

\begin{IEEEeqnarray}{rCl}
    \b{E} &=& q\left[\frac{(\hat{\b{n}}(t')-\b{\beta}(t'))(1-\b{\beta}(t')^2)}{K(t')^3R(t')^2}\right]_{R(t')=c(t-t')}\\
    && \negmedspace{} +\frac{q}{c}\left[\frac{\hat{\b{n}}(t')\times((\hat{\b{n}}(t')-\b{\beta}(t'))\times\dot{\b{\beta}}(t'))}{K(t')^3R(t')}\right]_{R(t')=c(t-t')},
\end{IEEEeqnarray}
\begin{equation}
    \b{B}=\left[\hat{\b{n}}\times\b{E}\right]_{R(t')=c(t-t')}.
\end{equation}

\begin{equation}
    \frac{\d P(t)}{\d \Omega} = \frac{c}{4\pi} R(t')^2 \left\lvert\b{E}(t')\right\rvert^2.
\end{equation}
$\d P(t)$为因$t'$时刻电荷的辐射, 观者在$t$时刻测得的单位时间通过$R(t')^2\d\Omega$的辐射.
求电荷处单位时间$\d\Omega$方向的能量$\d P(t')$. 能量守恒$\d P(t')\d t'=\d P(t)\d t$, $K(t')=\frac{\d t}{\d t'}$,
\begin{equation}
    \frac{\d P(t')}{\d \Omega} = \frac{c}{4\pi} K(t') R(t')^2 \left\lvert\b{E}(t')\right\rvert^2.
\end{equation}

非相对论, $K\simeq1$, $\hat{\b{n}}-\b{\beta}\simeq\hat{\b{n}}$,
\begin{equation}
    \b{E} = \frac{q}{cR}[\hat{\b{n}}\times(\hat{\b{n}}\times\dot{\b{\beta}})],
\end{equation}
\begin{equation}
    \frac{\d P}{\d \Omega} = \frac{q^2}{4\pi c}\dot{\beta}^2\sin^2\Theta_{\dot{\b{\beta}},\hat{\b{n}}}.
\end{equation}
