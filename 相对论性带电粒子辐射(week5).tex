\chapter{相对论性带电粒子辐射}\label{w5}

\begin{equation}
    \frac{\d P(t')}{\d \Omega} = \frac{c}{4\pi} K(t') R(t')^2 \left\lvert\b{E}(t')\right\rvert^2,
\end{equation}
\begin{equation}
    \b{E} =\frac{q}{c}\frac{\hat{\b{n}}(t')\times((\hat{\b{n}}(t')-\b{\beta}(t'))\times\dot{\b{\beta}}(t'))}{K(t')^3R(t')},
\end{equation}
\begin{equation}
    K(t')=1-\hat{\b{n}}(t')\cdot\b{\beta}(t').
\end{equation}

若$\hat{\b{n}}/\!/\b{\beta}$, 则$K\to0$, $\frac{\d P}{\d \Omega}\to\infty$, 速度方向极大.

$K=2K_\text{min}\to\theta\simeq\frac{1}{\gamma}:=\sqrt{1-\b{\beta}^2}$\footnote{此处$\theta$是观者系中粒子运动方向和观者方向的夹角.}.

粒子静系中$\theta=\frac{\pi}{2}$出射的光子\footnote{此处$\theta$是粒子静系中光子出射方向和粒子运动方向的夹角.}, 观者看来$\theta\simeq\frac{1}{\gamma}$.

$\b{\beta}/\!/\dot{\b{\beta}}$,
\begin{equation}
    \frac{\d P}{\d \Omega} = \frac{q^2}{4\pi c} \frac{\dot{\b{\beta}}^2\sin^2\theta}{(1-\left\lvert\b{\beta}\right\rvert\cos\theta)^5},
\end{equation}
$\theta_\text{max}=\arccos\frac{\sqrt{15\left\lvert\b{\beta}\right\rvert^2+1}-1}{3\left\lvert\b{\beta}\right\rvert}$\footnote{我算的, 不知对不对.}.

$\b{\beta}\bot\dot{\b{\beta}}$,
\begin{equation}
    \frac{\d P}{\d \Omega} = \frac{q^2}{4\pi c}\dot{\b{\beta}}^2
    \left[
        \frac{\gamma^2(1-\left\lvert\b{\beta}\right\rvert\cos\theta)^2
        -\sin^2\theta\cos^2\phi}{\gamma^2(1-\left\lvert\b{\beta}\right\rvert\cos\theta)^5}
    \right],
\end{equation}
其中$\phi$坐标系为: 以$\dot{\b{\beta}}$方向为$x$轴正向, 以$\b{\beta}$方向为$z$轴正向.

粒子系$\tilde{X}_\mu$, 实验室系${X}_\mu$.
\begin{equation}
    \d \tilde{W}=\frac{2q^2}{3c^3}\dot{v}^2\d \tilde{t},
\end{equation}
\begin{equation}
    \frac{\d \tilde{U}_\mu}{c\d \tau}=\left(
        \frac{1}{c^2}\frac{\d \tilde{v}_1}{\d s},
        \frac{1}{c^2}\frac{\d \tilde{v}_2}{\d s},
        \frac{1}{c^2}\frac{\d \tilde{v}_3}{\d s},
        0
    \right)
\end{equation}
\begin{equation}
    \d \tilde{P}_4=\frac{2q^2}{3c}\left(\frac{\d \tilde{U}_\mu}{c\d \tau}\right)^2\d \tilde{X}_4,
\end{equation}
\begin{equation}
    \d \tilde{P}_\alpha=\frac{2q^2}{3c}\left(\frac{\d \tilde{U}_\mu}{c\d \tau}\right)^2\d \tilde{X}_\alpha,
\end{equation}
\begin{equation}
    \d {P}_\alpha=\frac{2q^2}{3c}\left(\frac{\d {U}_\mu}{c\d \tau}\right)^2\d {X}_\alpha,
\end{equation}
\begin{equation}
    \d {P}_4=\frac{2q^2}{3c}\left(\frac{\d {U}_\mu}{c\d \tau}\right)^2\d {X}_4,
\end{equation}
\begin{equation}
    \d \tau=\gamma^{-2}\d t,
\end{equation}
\begin{equation}
    P=\frac{2q^2}{3c}\gamma^2\left(\frac{\d {U}_\mu}{\d t}\right)^2,
\end{equation}
\begin{equation}
    {U}_\mu=\left(
        \frac{\gamma}{c}v_1,
        \frac{\gamma}{c}v_2,
        \frac{\gamma}{c}v_3,
        i\gamma
    \right)
\end{equation}
