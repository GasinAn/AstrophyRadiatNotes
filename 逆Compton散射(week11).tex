\chapter{逆Compton散射}

\section{Thomson散射}

Thomson散射: 电子被电磁波强迫振动.

假设入射完全线偏振, $v$小, 忽略磁场, 电子成偶极子. 置偶极子于$z$轴,
\begin{equation}
    \frac{\d P}{\d \Omega}=\left(\frac{c}{4\pi}E^2\right)\left(\frac{e^2}{m_ec^2}\right)^2\sin^2\theta.
\end{equation}
微分散射截面
\begin{equation}
    \frac{\d \sigma}{\d \Omega}:=\left(\frac{e^2}{m_ec^2}\right)^2\sin^2\theta,
\end{equation}
散射截面
\begin{equation}
    \sigma=\frac{8\pi}{3}\left(\frac{e^2}{m_ec^2}\right)^2=\frac{8}{3}\pi r_e^2
\end{equation}
对无偏振入射也恰好成立, 无偏振入射散射后会偏振.

\section{Compton散射}

掏出QED, 得Klein-Nishina公式
\begin{equation}
    \frac{\d \sigma}{\d \Omega}=\frac{1}{2}\left(\frac{\nu_\text{出}}{\nu_\text{入}}\right)^2\left(\frac{\nu_\text{入}}{\nu_\text{出}}+\frac{\nu_\text{出}}{\nu_\text{入}}-\sin^2\theta\right)r_e^2,
\end{equation}
注意$\theta$是散射角(入射方向和散射方向的夹角), 且是无偏振入射. 入射能量大, $\frac{\d \sigma}{\d \Omega}$堆向$\theta=0$, 光子几乎不会被散射.

\section{逆Compton散射}

电子系中有Compton散射(或Thomson散射), 然后转换参考系, 结果是光子能量变$\gamma^2$倍, 且电子总认为光子从正面来, 产生各向同性散射, 但观者认为光子沿电子运动方向射出.

各向同性单色光辐射谱: 计$x=\frac{\nu}{4\gamma^2\nu_\text{入}}$, 峰值$x\simeq0.61$,截止$x=1$, 低频$P(\nu)\propto\nu$ 
