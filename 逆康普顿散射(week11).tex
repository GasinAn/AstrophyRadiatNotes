\chapter{逆康普顿散射}

\section{Thomson散射}

Thomson散射: 电子被电磁波强迫振动.

假设入射完全线偏振, $v$小, 忽略磁场, 电子成偶极子. 置偶极子于$z$轴,
\begin{equation}
    \frac{\d P}{\d \Omega}=\left(\frac{c}{4\pi}E^2\right)\left(\frac{e^2}{m_ec^2}\right)^2\sin^2\theta.
\end{equation}
微分散射截面
\begin{equation}
    \frac{\d \sigma}{\d \Omega}:=\left(\frac{e^2}{m_ec^2}\right)^2\sin^2\theta,
\end{equation}
散射截面
\begin{equation}
    \sigma=\frac{8\pi}{3}\left(\frac{e^2}{m_ec^2}\right)^2=\frac{8}{3}\pi r_e^2
\end{equation}
对无偏振入射也恰好成立.

线偏振散射后$\b{E}$在入射$\b{E}$和出射$\b{n}$面内且垂直于$\b{n}$.

\section{Compton散射}

掏出QED, 得Klein-Nishina公式
\begin{equation}
    \frac{\d \sigma}{\d \Omega}=\frac{1}{2}\left(\frac{\nu_\text{出}}{\nu_\text{入}}\right)^2\left(\frac{\nu_\text{入}}{\nu_\text{出}}+\frac{\nu_\text{出}}{\nu_\text{入}}-\sin^2\theta\right)r_e^2,
\end{equation}
注意$\theta$是散射角(入射方向和散射方向的夹角), 且是无偏振入射. 入射能量大, $\frac{\d \sigma}{\d \Omega}$堆向$\theta=0$, 光子几乎不会被散射.

\section{逆Compton散射}

电子系中有Compton散射(或Thomson散射), 然后转换参考系, 结果是光子能量变$\gamma^2$倍, 且电子总认为光子从正面来, 产生各向同性散射, 但观者认为光子沿电子运动方向射出.

单电子, 各向同性光子,
\begin{equation}
    P(\nu)=8\pi r_e^2hc\int f(x(\nu,\gamma,\nu_\text{入}))n_\text{ph}(\nu_\text{入})\d\nu_\text{入},
\end{equation}
$x:=\nu/4\gamma^2\nu_\text{入}$, $0<x<1$, $f$峰$x=0.61$, 低频$\propto\nu$.

电子系$N(\gamma)$, 
\begin{equation}
    j(\nu)=8\pi r_e^2hc\iint N(\gamma)f(x(\nu,\gamma,\nu_\text{入}))n_\text{ph}(\nu_\text{入})\,\d\nu_\text{入}\d\gamma.
\end{equation}

电子系$N(\gamma)\propto\gamma^{-n}$, $\gamma_1<\gamma<\gamma_2$, $\nu\gg\nu_\text{入}$\footnote{$x>1\to\nu>4\pi\gamma_{1}^2\nu_\text{入}$.}, $x=1\to\gamma_\text{min}\gg\gamma_1$, $\gamma_2=\infty$, $j(\nu)\propto\nu^{-(n-1)/2}$, 同同步辐射.

光子$n_\text{ph}(\nu_\text{入})\propto\nu_\text{入}^{-p}$, $\nu_\text{入1}<\nu_\text{入}<\nu_\text{入2}$, $\nu>4\gamma^2\nu_\text{入1}$, $\nu_\text{入2}=\infty$, $j(\nu)\propto\nu^{-(p-1)}$. 注意$j(\nu_\text{入})\propto\nu_\text{入}^{-(p-1)}$.

电子光子都幂律, 分情况.
