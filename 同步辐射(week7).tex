\chapter{同步辐射}

\section{回旋辐射}\label{cyclotron}

$v_\parallel=v\cos\alpha$, $v\perp=v\sin\alpha$, $\alpha$称为投射角.

非相对论,
\begin{equation}
    P=\frac{2e^2\dot{v}^2}{3c}.
\end{equation}
又有$m_e\dot{\b{v}}=-\frac{e}{c}(\b{v}\times\b{B})$, 所以
\begin{equation}
    P=\frac{2}{3}\frac{e^4}{m_e^2c^5}v^2B^2\sin^2\alpha.
\end{equation}
电子经典半径为$r_e=\frac{e^2}{m_ec^2}$\footnote{令$m_ec^2=\frac{e^2}{r_e}$.}, 则
\begin{equation}
    P=\frac{2}{3c}r_e^2v^2B^2\sin^2\alpha=\frac{2r_e^2c}{3}\beta^2B^2\sin^2\alpha.
\end{equation}
假设电子分布各项同性, 则平均总功率
\begin{align}
    \bar{P}&=\frac{2r_e^2c}{3}\beta^2B^2\frac{\iint\sin^2\alpha\sin\alpha\,\d\alpha\,\d\phi}{4\pi}\\
    &=\frac{2r_e^2c}{3}\beta^2B^2\frac{\int\sin^3\alpha\,\d\alpha}{2}\\
    &=\frac{2r_e^2c}{3}\beta^2B^2\frac{2}{3}\\
    &=\frac{4}{9}r_e^2c\beta^2B^2.
\end{align}
$P$和$\bar{P}$正比于电子动能($\beta^2$)和磁场能量密度($B^2$).

周期运动,
\begin{equation}
    E(t)=\sum_{-\infty}^{\infty} E_se^{-is\omega_0 t},
\end{equation}
\begin{equation}
    E_s=\frac{1}{T}\int_0^T E(t)e^{is\omega_0 t}\d t.
\end{equation}

对$\frac{\d W}{\d \Omega}:=\int_0^T\frac{\d P(t)}{\d \Omega}\,\d t$作Fourier分析. 一样假定$R\simeq\text{const}$, 由Parseval定理,
\begin{equation}
    \frac{\d\bar{P}_s}{\d\Omega}:=\frac{1}{T}\frac{\d W_s}{\d\Omega}=\frac{c}{2\pi}R^2
    \left\lvert\b{E}_s\right\rvert^2.
\end{equation}
\begin{equation}
    \b{E}_s=\frac{1}{T}\int_0^T \b{E}(t')e^{is\omega_0 t}\d t.
\end{equation}
由$t'=t-\frac{R(t')}{c}$, $\frac{\d t}{\d t'}=K(t')$,
\begin{equation}
    \b{E}_s=\frac{1}{T}\int K(t')\b{E}(t')e^{i\omega \left[t'+\frac{R(t')}{c}\right]}\d t',
\end{equation}
$R(t')=\left\lvert\b{r}-\b{r}_q(t')\right\rvert\simeq\left\lvert\b{r}\right\rvert-\frac{\b{r}}{\left\lvert\b{r}\right\rvert}\cdot\b{r}_q(t')=\left\lvert\b{r}\right\rvert-\hat{\b{n}}\cdot\b{r}_q(t')$, 得
\begin{equation}
    \left\lvert\b{E}_s\right\rvert^2=\left\lvert\frac{1}{T}\int K(t')\b{E}(t')e^{i\omega \left[t'-\frac{\hat{\b{n}}\cdot\b{r}_q(t')}{c}\right]}\d t'\right\rvert^2,
\end{equation}
非相对论,
\begin{equation}
    \frac{\d\bar{P}_s}{\d\Omega}
    =\frac{q^2\left\lvert{\ddot{\b{r}}}_q\right\rvert_s^2\sin^2
    \Theta_{\ddot{\b{r}}_q,\hat{\b{n}}}}{2\pi c^3}
    =\frac{s^4\omega_0^4q^2\left\lvert{\b{r}}_q\right\rvert_s^2\sin^2
    \Theta_{\ddot{\b{r}}_q,\hat{\b{n}}}}{2\pi c^3}.
\end{equation}

偶极子: $x$轴和$y$轴$\b{E}$沿$z$轴, $z$轴$\b{E}=\b{0}$.

由对称性, 可令电子在$xOy$面内, 观者在$xOz$面内, 然后可得($\nu_0=\frac{\omega_0}{2\pi}$, $\omega_0=\frac{1}{\gamma}\omega_L=\frac{1}{\gamma}\frac{eB}{m_ec}$)
\begin{equation}
    \frac{\d\bar{P}_s}{\d\Omega}
    =\frac{2\pi e^2s^2\nu_0^2}{c}
    \left[\cot^2\theta J_s(s\beta\sin\theta)^2+\beta^2J_s'(s\beta\sin\theta)^2\right],
\end{equation}
其中Bessel函数
\begin{equation}
    J_s(x):=\frac{1}{2\pi}\int_0^{2\pi}e^{i(su-x\sin u)}\d u.
\end{equation}
积分得
\begin{equation}
    \bar{P}_s
    =\frac{8\pi^2e^2s^2\gamma^{-2}\nu_L^2}{c\beta}
    \left[s\beta^2J_{2s}'(2s\beta)-s^2\gamma^{-2}\int_0^\beta J_{2s}(2su)\d u\right],
\end{equation}
非相对论, $\beta\ll1$, $s\beta\ll1$, 展开得
\begin{equation}
    \bar{P}_s\simeq\left(\frac{8\pi^2e^2\nu_L^2}{c}\right)
    \frac{(s+1)s^{2s+1}}{(2s+1)!}\beta^{2s}.
\end{equation}
可见$\bar{P}_{s+1}/\bar{P}_s\sim\beta^2\ll1$.

\section{同步辐射}

同样回旋运动, 令$m_e\to\gamma m_e$, 得$\omega_0=\frac{1}{\gamma}\omega_L=\frac{eB}{\gamma m_ec}$, 回旋半径非常大, 接近直线运动.
\begin{equation}
    P=\frac{2e^2\gamma^4\dot{v}^2}{3c}.
\end{equation}
\begin{equation}
    P=\frac{2r_e^2c}{3}\gamma^2\beta^2B^2\sin^2\alpha.
\end{equation}
\begin{equation}
    \bar{P}=\frac{4}{9}r_e^2c\gamma^2\beta^2B^2.
\end{equation}
冷却时间$t_\text{cool}:=\gamma m_ec^2/\bar{P}$.

仍然有第\ref{w5}章的$\frac{\d P}{\d\Omega}$和\ref{cyclotron}节的
\begin{equation}
    \frac{\d\bar{P}_s}{\d\Omega}
    =\frac{2\pi e^2s^2\nu_0^2}{c}
    \left[\cot^2\theta J_s(s\beta\sin\theta)^2+\beta^2J_s'(s\beta\sin\theta)^2\right],
\end{equation}
\begin{equation}
    \bar{P}_s
    =\frac{8\pi^2e^2s^2\nu_0^2}{c\beta}
    \left[s\beta^2J_{2s}'(2s\beta)-s^2\gamma^{-2}\int_0^\beta J_{2s}(2su)\d u\right].
\end{equation}
因为$\Delta\nu_0/\nu_0\ll1$, 已成连续谱.

直接求连续谱. $\omega_c:=\frac{3}{2}\gamma^2\omega_L\sin\alpha$, 最大值$\omega\approx0.29\omega_c$, 低频$\propto\omega^{1/3}$, 高频指数下降, 用$\ln\omega$画图可见高频截断.

\section{幂律分布电子的集体辐射}

常认为电子能量分布是幂律的, 即
\begin{equation}
    N(\gamma)=C\gamma^{-p},\quad\gamma\in[1,\infty),
\end{equation}
则
\begin{equation}
    P_\text{tot}(\omega)=C\int_{\gamma_1}^{\gamma_2}P(\omega)\gamma^{-p}\,\d\gamma\propto\int_{\gamma_1}^{\gamma_2}F\left(\frac{\omega}{\omega_c}\right)\gamma^{-p}\,\d\gamma.
\end{equation}
把积分变量变成$x:=\omega/\omega_c$, 由$\omega_c\propto\gamma^2$得
\begin{equation}
    P_\text{tot}(\omega)\propto\omega^{-(p-1)/2}\int_{x_1}^{x_2}F\left(x\right)x^{(p-3)/2}\,\d x\propto\omega^{-s},
\end{equation}
其中$s:=(p-1)/2$.

\section{同步自吸收}

$\alpha(\omega)\propto\omega^{-(p+4)/2}$, $S(\omega)=\frac{j(\omega)}{\alpha(\omega)}=\frac{P_\text{tot}(\omega)}{4\pi}\alpha(\omega)\propto\omega^{5/2}$. 低频$\alpha$大, 光学厚, $I(\omega)\approx S(\omega)\propto\omega^{5/2}$, 高频$I(\omega)\propto\omega^{-s}$, 分界点$\tau(\omega)=1$.

光学厚非黑体谱的原因是电子分布非热(幂律).

\section{曲率辐射}

因为运动类似, 所以单粒子谱型一样. 但$\omega_0\simeq\frac{c}{\rho}$ (其中$\rho$为曲率半径), 所以$\omega_c\propto\gamma^3$, 低频端$P(\omega)\propto\omega^{1/3}$与$\gamma$无关, 所以$P_\text{tot}(\omega)\propto P(\omega)\propto\omega^{1/3}$.
